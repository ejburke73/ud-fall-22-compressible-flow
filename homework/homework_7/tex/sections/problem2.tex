\documentclass[../main.tex]{subfiles}

\begin{document}

\problem{2}

\problempart{a}

Write out a detailed methodology for the solution of flow around an axisymmetric cone with a sharp nosetip.

\problempart{b}
Plot shock half angle \(\theta_s\) vs. cone half angle \(\theta_c\) for \(M_\infty=1.25,\,2.0,\,6.0,\,\textrm{and}\,10.0\)

\problempart{c}

Repeat part (b) but include results for a wedge with the same shock angles. 
Explain why you observe these differences between cones and wedges.

\problempart{d}

Plot the Mach number at the cone surface, \(M_c\), vs. cone half angle \(\theta_c\) for \(M_\infty=1.25,\,2.0,\,6.0,\,\textrm{and}\,10.0\).

\problempart{e}

Assuming an angle of attack of \(0^\circ\) and \(\gamma=1.4\), use the given flight data to recreate the Taylor-Maccoll plot from the left plot in Fig. 22.

\problempart{f}

Describe how the Taylor-Maccoll code could be used to estimate the angle of attack of the flight vehicle.

\end{document}