\documentclass[../main.tex]{subfiles}

\begin{document}

\problem{1}

\problempart{a}

Calculate \(p_\infty\) and \(p_0\) at the exit of the nozzle before the NS reaches it.

\problempart{b}

Calculate \(p_\infty\) and \(p_0\) at the exit of the nozzle after the shock passes and before the contact surface arrives.

\problempart{c}

If you had a blunt wind-tunnel model at the exit plane of the nozzle, what \(p_0\) would it feel at the blunted tip surface in the state-2 flow?
Briefly justify your answer.

\problempart{d}

How long will it take for the NS to reach the probe from the time the valve opens?
Generally speaking, what does this mean for your test-section data if your instrumentation within it is triggered to start collecting data right when the valve opens?

\problempart{e}

How much time elapses between the passage of the NS and of the contact surface?

\problempart{f}

In actuality, given the values in the Normal-Shock part of the problem, the expansion wave would propagate downstream initially.
So, for this expansion-wave part of the problem re-do the necessary calculations from above for \(p_0 = 1000 \, \unit{\kilo\pascal}\) and \(T_0 = 500\,\unit{\kelvin}\) and the back air at \(p=101\,\unit{\kilo\pascal}\) and \(T_0 = 295\,\unit{\kelvin}\). 

\problempart{g}

Use the method of characteristics to plot \(p/p_4\) and \(T/T_4\) as a function of time at the end wall of the driver tube.
Do this for a few different number of characteristics.
How many characteristics seems like enough?

\problempart{h}

Plot the characteristics from the valve to the end wall and back.

\problempart{i}

How long does it take for the head to get back to the location of the valve? 
If you approximated this time assuming a constant head speed equal to the local speed of sound (as done in the paper), would you expect it to be different than how you calculated if? 
If so would it be larger or smaller?

\end{document}