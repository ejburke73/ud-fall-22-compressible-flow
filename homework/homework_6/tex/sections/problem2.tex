\documentclass[../main.tex]{subfiles}

\begin{document}

\problem{2}

\givens{}

\(p_0=4000\,\unit{\kilo\pascal}\)\\
\(T_0 = 500\,\unit{\kelvin}\)\\
\(D_{throat, inviscid} = 4.114\,\unit{in.} = 0.1045\,\unit{\meter}\)\\
\(D_{throat, real} = 3.71\,\unit{in.} = 0.0942\,\unit{\meter}\)\\
\(M_e=6\)

\assumptions{}
Flow is isentropic everywhere outside of any shock waves that are present.
Steady, inviscid, quasi-1D flow through nozzle. 
Upstream reservoir is large enough to assume that static conditions are equal to total conditions. 
The nozzle is choked, i.e., \(M_{throat}=1\).
\(\gamma=1.4,\,R=287\,\unit{\joule/\kilogram\cdot\kelvin}\).

\textit{Note: All calculations performed in Python, see Appendix \ref{Problem2Python}.}

\begin{enumerate}[label=(\alph*)]

    \item Mass flowrate through a choked nozzle is given by the following equation derived in Problem 1:
    
    \[
        \dot{m} = \frac{p_0 A^*}{\sqrt{T_0}} \sqrt{\frac{\gamma}{R} \left({\frac{2}{\gamma+1}}\right)^{\frac{(\gamma+1)}{(\gamma-1)}}}
    \]

    \(p_0\) and \(T_0\) are given and \(A^*\) is easily calculated:

    \[
        A^* = \frac{\pi D_{throat}^2}{4} =  0.0086 \, \unit{\meter\squared}
    \]

    \[
        \boxed{
        \dot{m} = 62.01 \,\unit{\kilogram/\second}
        }
    \]

    \item The Area-Mach relationship is given by the following equation:
    
    \[
        {\left({\frac{A}{A^*}}\right)}^2
        =
        \frac{1}{M^2}
        {\left[{
            \frac{2}{\gamma+1}
            \left({1 + \frac{\gamma-1}{2}M^2}\right)
        }\right]}^{(\gamma+1)/(\gamma-1)}
    \]

    With a known throat area (\(A^*\)) and exit Mach (\(M_e=6\)) it is simple to calculate exit area, \(A_e\):

    \[
        A_e
        =
        \sqrt{
        \frac{{A^*}^2}{M^2}
        {\left[{
            \frac{2}{\gamma+1}
            \left({1 + \frac{\gamma-1}{2}M^2}\right)
        }\right]}^{(\gamma+1)/(\gamma-1)}
        }
    \]

    \[
        \boxed{A_e =  0.4560\,\unit{\meter\squared} = 706.8 \, \unit{in\squared}}
    \]

    The calculated exit area of \(706.8\,\unit{in\squared}\) compares well to the exit dimensions in the conference paper which list an exit diameter of \(D_{e}=30\,\unit{in}\).
    The calculated exiit diameter is \(D_e = 29.99\,\unit{in}\) which is well within an acceptable margin of error.

    \item The design back pressure, \(p_{b,design}\), and temperature, \(T_{b,design}\) are the static conditions associated with \(M_e=6\) exit flow assuming no shocks in the nozzle.
    Using isentropic relations and treating the reservoir conditions as total conditions for the flow, the design point conditions at the exit of the nozzle are as follows:

    \[
        \boxed{p_{b,design} = 2533.45\,\unit{\pascal} \qquad T_{b,design} = 60.97\,\unit{\kelvin}}
    \]    
    
    \item The lowest back pressure for which there is only subsonic flow in the nozzle can be determined by using the critical back pressure ratio, \(p^*/p_0\).
    
    \[
        \frac{p^*}{p_0} = {\left(1 + \frac{\gamma-1}{2}\right)}^{-\frac{\gamma}{\gamma-1}} = 0.52828
    \]

    Setting \(p_b=p^*\) yields the critical value of back pressure.
    This is the back pressure for which the throat reaches sonic conditions, so any value of back pressure slightly above this (represented by the term \(\varepsilon\))  will yield only subsonic flow in the nozzle.

    \[
        \boxed{p_b = 2113127\,\unit{\kilo\pascal} + \varepsilon}
    \]

    \item The back pressure for which there is a normal shock at the nozzle exit plane is given by the value of pressure associated with the design exit Mach goingi through a normal shock wave.
    We use normal shock relations to determine the post-normal shock static pressure for \(M_e=6\) flow.

    \[
        \boxed{p_{b} = 105982 \, \unit{\pascal}}
    \]  

    \item The critical value of back pressure below which there are no shock waves (OS or EW) in the nozzle is the same as the pressure for which the normal shock stands at the exit plane.
    There is a very exact value of pressure that will keep the shock at the nozzle exit, pressures on either side of this value can push shock waves into or out of the nozzle.    
    
    \[
        \boxed{p_{b} < 105982 \, \unit{\pascal}}
    \]

    \item The range of back pressures for which there are oblique shock waves in the nozzle exhaust is limited on the high end by the exit-plane normal shock value and on the lower end by the design exit pressure for \(M_e=6\).
    In this range, the exit flow has a lower static pressure than the downstream region and must go through an oblique shock to be in equilibrium.
    
    \[
        \boxed{2533.45\,\unit{\pascal} < p_{b} < 105982 \, \unit{\pascal}}
    \]

    \item The range of back pressures for which there are expansions waves is limited on the high end by the design exit pressure, \(p_e\), and has no limit on the lower end.
    In this range, the exit flow has a higher static pressure than the downstream region and must go through an expansion wave to be in equilibrium.

    \[
        \boxed{p_{b} < 2533.45\,\unit{\pascal}}
    \]

    \item To find the back pressure for which a normal shock wave occurs in the divergent section of the nozzle at the point where the cross-sectional area is equal to the average of the throat and exit planes, we start by calculating the average area.
    
    \[
        A_{avg} = 0.2323 \, \unit{\meter\squared}
    \]

    Using the Area-Mach number relation and a numerical solver, we determine the supersonic velocity at this point.

    \[
        M_{avg} = 5.1  
    \]
    
    Finally, we use normal shock relations to determine the post-normal shock pressure associated with this flow, which is equivalent to the back pressure that would hold a normal shock at this point in the nozzle.

    \[
        \boxed{p_b = 203040\, \unit{\pascal}}  
    \]

    \item To calculate the time until there is a normal shock at the exit plane of the nozzle, we must determine the pressures in the receiving tanks assuming a constant choked mass flow rate.
    We assume that transient effects are negligible and the temperature in the receiving tanks is a constant \(T=295\,\unit{\kelvin}\).
    We also make the assumption that the receiving tanks are large enough that the air is essentially still and static conditions are equal to total conditions.
    For a constant tank temperature, the back pressure will be a function solely of the density of the air in the receiving tanks.
    Using the ideal gas law, \(p = \rho R T\), we have a relationship between back pressure and the volume of air in the tank. 
    Using the choked mass flow rate and the volume of the tanks, we can determine the density of the air in the receiving tanks at any time \(t\).

    \[
        \dot{m} = 62.01 \,\unit{\kilogram/\second}
    \]

    \[
        \volume_{tanks} = 4000 \, \unit{gal} = 15.14165\,\unit{\meter\cubed}
    \]

    \[
        p_b(t) = \frac{\dot{m}}{\volume_{tanks}}R T t
    \]

    Knowing the value of back pressure we want to solve for (\(p_b = 105982 \, \unit{\pascal}\)) we can easily solve for the time at which the shock will exist at the nozzle exit.

    \[
        t_{NS} = \frac{p_b \volume_{tanks}}{\dot{m} R T}  
    \]

    \[
        \boxed{t_{NS} = 0.3056\,\unit{\second}}
    \]

    The tunnel's diffuser would increase the static pressure of the downstream flow heading into the receiving tanks.
    Assuming a constant receiving tank temperature, the increased static pressure would also increase the density in the receiving tanks, reducing the time it takes for a normal shock to appear.
    However, the normal shock still has to travel upstream (a transient effect) and reach the test section before the high-quality flow in the test section is no longer useful.

    \item To determine the length of the driver tubes, we must first find calculate the fluid velocity during the time frame of interest.
    Assuming that conditions in the driver tube are equal to the stagnation conditions of the upstream reservoir, we first calculate the fluid density in the driver tube.

    \[
        \rho_{tube} = \frac{P_0}{RT_0} = 27.87 \, \unit{\kilogram/\meter\cubed}
    \]

    Next, the amount of mass used during the time frame of interest is found using the known mass flow and time.

    \[
        m_{used} = \dot{m} t = 18.95 \, \unit{\kilogram}  
    \]

    The volume of air at this density associated with the known mass is then calculated.

    \[
        \volume = \frac{m_{used}}{\rho_{tube}} = 0.68 \, \unit{\meter\cubed}
    \]

    From the paper, the driver tubes have an inner diameter of \(D=9.75\,\unit{inch}=0.24765\,\unit{\meter}\).
    The area of the tube is easily calculated.

    \[
        A_{tube} = \frac{\pi D_{tube}^2}{4} = 0.0482 \, \unit{\meter\squared}  
    \]

    With a known volume and area, we can calculate the length of driver tube used during the time frame of interest.

    \[
        L_{tube} = \frac{\volume}{A_{tube}}  
    \]

    \[
        \boxed{L_{tube} = 14.12 \, \unit{\meter} = 46.31 \, \unit{ft}}
    \]

    With a total internal driver tube length of 82 feet, this value is within reason.

\end{enumerate}

\end{document}