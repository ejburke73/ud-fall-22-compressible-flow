\documentclass[../main.tex]{subfiles}

\begin{document}

\problem{1}

Starting with \(\dot{m} = \rho u A\), prove that the mass flowrate through an isentropic choked nozzle can be written in the form:

\[
    \dot{m} = \frac{p_0 A^*}{\sqrt{T_0}} \sqrt{\frac{\gamma}{R} \left({\frac{2}{\gamma+1}}\right)^{\frac{(\gamma+1)}{(\gamma-1)}}}
\]

\assumptions{}
Isentropic flow through a nozzle with a choked (sonic) throat.

\solution{}

The mass flow at a given cross section in a quasi 1-D flow is given by:

\[
    \dot{m} = \rho u A
\]

Choosing the throat of a choked nozzle as the point of interest, we replace the conditions with sonic conditions, denoted by \(*\) and indicating the flow property at the location where \(M=1\).

\[
    \dot{m} = \rho^* u^* A^*
\]

In order to cast this in terms of properties that are more easily known ahead of time, we identify relationships involving the total conditions of the flow, beginning with \(\rho^*\).
Using the ideal gas law, we can cast the sonic density in terms of pressure and temperature:

\[
    p^* = \rho^* R T^*
\]

\[
    \rho^* = \frac{p^*}{RT^*}
\]

Now, we find relationships for \(p^*\) and \(T^*\).

Beginning with the isentropic relationship between total and static pressure:

\[
    \frac{p_0}{p} = \left({
        1 + \frac{\gamma-1}{2}M^2
    }\right)^{\frac{\gamma}{\gamma-1}}
\]

Setting \(M=1\):

\[
    \frac{p_0}{p^*} = \left({
        1 + \frac{\gamma-1}{2}
    }\right)^{\frac{\gamma}{\gamma-1}}
\]

\[
    p^* = \frac{p_0}{\left({
        \frac{\gamma+1}{2}
    }\right)^{\frac{\gamma}{\gamma-1}}}
\]

For temperature:

\[
    \frac{T_0}{T} = \left({
        1 + \frac{\gamma-1}{2}M^2
    }\right)
\]

Setting \(M=1\):

\[
    \frac{T_0}{T^*} = \left({
        1 + \frac{\gamma-1}{2}
    }\right)
\]

\[
    T^* = \frac{T_0}{\left({
        \frac{\gamma+1}{2}
    }\right)}
\]

Substituting into the ideal gas equation yields an expression for \(\rho^*\) in terms of \(p_0,\, T_0,\, R,\,\textrm{and}\,\gamma\):

\[
    \rho^* =  
    \frac{p_0}{\left({
        \frac{\gamma+1}{2}
    }\right)^{\frac{\gamma}{\gamma-1}}}
    \frac{\left({
        \frac{\gamma+1}{2}
    }\right)}{T_0}
    \frac{1}{R}
\]

Next, we examine the sonic velocity term, \(u^*\).
Noting that for choked flow \(M=1\), we observe that the flow velocity must be equal to the speed of sound, \(a\).

\[
    M = \frac{u^*}{a^*} = 1 \rightarrow u^* = a^*
\]

\[
    a^* = \sqrt{\gamma R T^*}  
\]

Subsituting our known equation for \(T^*\):

\[
    a^* = \sqrt{\frac{\gamma RT_0}{\left({
        \frac{\gamma+1}{2}
    }\right)}}  
\]

Substituting everything back into the original mass flow equation:

\[
    \dot{m} = 
    \frac{p_0}{\left({
        \frac{\gamma+1}{2}
    }\right)^{\frac{\gamma}{\gamma-1}}}
    \frac{\left({
        \frac{\gamma+1}{2}
    }\right)}{T_0}
    \frac{1}{R}
    \sqrt{\frac{\gamma RT_0}{\left({
        \frac{\gamma+1}{2}
    }\right)}}  
    A^*
\]

Rearranging:

\[
    \dot{m}
    =
    \frac{p_0 A^*}{R T_0}
    \frac{\frac{\gamma+1}{2}}{\left({
        \frac{\gamma+1}{2}
    }\right)^{\frac{\gamma}{\gamma-1}}}
    \sqrt{\frac{\gamma RT_0}{\left({
        \frac{\gamma+1}{2}
    }\right)}}  
\]

\[
    \dot{m}
    =
    p_0 A^*
    \frac{1}{\left({
        \frac{\gamma+1}{2}
    }\right)^{\frac{\gamma}{\gamma-1}}}
    \sqrt{\frac{\gamma RT_0}{R^2 T_0^2}\frac{\left({
        \frac{\gamma+1}{2}
    }\right)^2}{\left({
        \frac{\gamma+1}{2}
    }\right)}}  
\]

\[
    \dot{m}
    =
    \frac{p_0 A^*}{\sqrt{T_0}}
    \frac{1}{\left({
        \frac{\gamma+1}{2}
    }\right)^{\frac{\gamma}{\gamma-1}}}
    \sqrt{\frac{\gamma}{R} \left({\frac{\gamma+1}{2}}\right)}
\]

\[
    \dot{m}
    =
    \frac{p_0 A^*}{\sqrt{T_0}}
    \sqrt{\frac{\gamma}{R} \left({\frac{\gamma+1}{2}}\right)^{1-\frac{2\gamma}{\gamma-1}}}
\]

\[
    \dot{m}
    =
    \frac{p_0 A^*}{\sqrt{T_0}}
    \sqrt{\frac{\gamma}{R} \left({\frac{\gamma+1}{2}}\right)^{\frac{-\gamma-1}{\gamma-1}}}
\]

\[
    \boxed{
    \dot{m} = \frac{p_0 A^*}{\sqrt{T_0}} \sqrt{\frac{\gamma}{R} \left({\frac{2}{\gamma+1}}\right)^{\frac{(\gamma+1)}{(\gamma-1)}}}
    }
    \]

\end{document}