\documentclass[../main.tex]{subfiles}

\begin{document}

\problem{4}

You will use this problem to derive the \textbf{differential} form of conservation of mass and momentum for \textbf{inviscid} flowfields.
You will do this by applying the integral form of these governing equations to a differential-element control volume.

\begin{enumerate}[label = (\alph*)]

    \item \textbf{Conservation of Mass:}
        \[
            \pdv{\rho}{t} + \pdv{(\rho u)}{x} + \pdv{(\rho v)}{y} + \pdv{(\rho w)}{z} = 0
        \]

        Watch the attached lecture from my undergraduate fluid-mechanics class and follow the steps to derive the differential form of the continuity equation.
        Be explicit about your assumptions and make the derivation your own.

    \item \textbf{(Inviscid) Conservation of Momentum:}
    
        \begin{align*}
            \rho \left({\pdv{u}{t} + u \pdv{u}{x} + v \pdv{u}{y} + w \pdv{u}{z}}\right) &= \rho g_x - \pdv{p}{x} + \mu \left({\pdv[2]{u}{x} + \pdv[2]{u}{y} + \pdv[2]{u}{z}}\right)\\
            \rho \left({\pdv{v}{t} + u \pdv{v}{x} + v \pdv{v}{y} + w \pdv{v}{z}}\right) &= \rho g_y - \pdv{p}{y} + \mu \left({\pdv[2]{v}{x} + \pdv[2]{v}{y} + \pdv[2]{v}{z}}\right)\\
            \rho \left({\pdv{w}{t} + u \pdv{w}{x} + v \pdv{w}{y} + w \pdv{w}{z}}\right) &= \rho g_z - \pdv{p}{z} + \mu \left({\pdv[2]{w}{x} + \pdv[2]{w}{y} + \pdv[2]{w}{z}}\right)\\
        \end{align*}

    \begin{enumerate}[label = (b\arabic*)]

        \item 
            Draw your differential element with coordinate system.
            Label each side \(\delta x, \delta y, \ \textrm{and} \ \delta z\), respectively.

        \item
            Label each of the six sides with the appropriate momentum values.
            Include the direction of the momentum with arrows.
            This process will be identical to what we did for Continuity, except now we will have an additional velocity (i.e., \(\vec{V}\)) term since we are talking about momentum.
            For example, the left-hand-side of the differential element should have \(\rho u \vec{V}\) flowing into it instead of just \(\rho u\).
            You can show this analysis on one single differential-element drawing, or three separate ones.

        \item
            Now, write out the left-hand-side of the general Integral Momentum Equation, state the relevant assumptions/simplifications based on your differential element control-volume drawing(s), and simplify each term as much as you can.

        \item
            Tabulate the influx and outflux of momentum for each face.
            Do not forget about the corresponding area terms.

        \item
            Populate your simplified Momentum Equation with these influx and outflux terms appropriately.

        \item
            Show how the three outflux terms cancel with the three influx terms to give:
            \[
                \left[{
                    \pdv{(\rho \vec{V})}{t} + \pdv{(\rho u \vec{V})}{x} + \pdv{(\rho v \vec{V})}{y} + \pdv{(\rho w \vec{V})}{z}
                }\right]
                \delta x \delta y \delta z
            \]
        
        \item
            Each individual term in Eq. can be broken up via the Chain Rule for derivatives, which states:
            \[
                \pdv{(ab)}{c} = a\pdv{(b)}{c} + b\pdv{(a)}{c}  
            \]
            Use this rule to break up each of the four partial derivatives in Eq. .
            For terms 2-4, treat the products $\rho u$, $\rho v$, and $\rho w$ as $a$ from Eq. , and $\vec{V}$ as $b$.
            That is, term 2 from Eq. can be split up as:
            \[
                \pdv{(\rho u \vec{V})}{x} = \rho u \pdv{(\vec{V})}{x} + \vec{V}\pdv{(\rho u)}{x}   
            \]
            Your updated Eq. should now have eight terms inside the brackets.
            Four of these terms are the Differential Continuity Equation and their combination can therefore be set to zero.
            Show this, and show that the neq equation is:
            \[
                \left[{
                    \rho\pdv{(\vec{V})}{t} + \rho u\pdv{(\vec{V})}{x} + \rho v\pdv{( \vec{V})}{y} + \rho w \pdv{(\vec{V})}{z}
                }\right]
                \delta x \delta y \delta z
            \]
            Woohoo, this is the final form of the left-hand-side of the Differential Momentum Equation!

        \item
            Now calculate the forces due to gravity and pressure on the differential element and add them to the right hand side of your differential momentum equation.
            Note: for pressure, you may allow for pressure to change in each direction across your differential element via a Taylor-Series-Expansion first-order approximation like you've done in other parts of the problem.
        
        \item
            Put everything together and write the final form appropriately for the $x, y, \textrm{and} \ z$ directions to give the $x, y, \textrm{and} \  z$ differential momentum equations. 
        
    \end{enumerate}
    
\end{enumerate}

\end{document}