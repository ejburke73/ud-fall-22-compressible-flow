\documentclass[../main.tex]{subfiles}

\begin{document}

\problem{1}

The final Reynolds Transport Theorem we derived in class looked like:

\begin{equation*}
    \frac{\dd B_{sys}}{\dd t} = %
    \frac{\dd (mb)_{sys}}{\dd t} = %
    \pdv{t} \int_{CV} \rho b \dd V +%
    \int_{CS,out} b \rho \lvert{\vec{V_n}}\rvert \dd A -%
    \int_{CS,out} b \rho \lvert{\vec{V_n}}\rvert \dd A
\end{equation*}

\begin{enumerate}[label = (\alph*)]

    \item In your own words, describe what each of the three terms on the right-hand-side of the equation mean related to an arbitrary fluid extensive property, $B$.

    \item If our problem was in the $x-y-z$ space, how would you represent the integrals \(\int_{CV} \dd V\) and \(\int_{CS} \dd A\) in terms of triple and double integrals, respectively?

    \item Why are the last two terms integral terms?

    \item What does the subscript ``n'' mean for the last two terms? Why do we need that there?
   
    \item Why do we need the absolute magnitude signs around the \(\vec{V}_n\) terms?
   
    \item Why is the derivative with-respect-to \(t\) a partial derivative?
   
    \item Explain to a classmate how our 
        \[
            \int_{CS,out} b \rho \lvert{\vec{V_n}}\rvert \dd A -%
            \int_{CS,out} b \rho \lvert{\vec{V_n}}\rvert \dd A
        \]
        term is equivalent to

        \[
            \int_{CS} b \rho \boldsymbol{V} \cdot \hat{\boldsymbol{n}} \dd A \, ,
        \]

        which is equivalent to

        \[
            \int_{CS} b \rho \vec{V} \cdot \dd \vec{A} \, .
        \]

        Be sure to explain the different math concepts. You may find it easier to ``explain'' by using a simple control-volume problem as an illustration.
        
\end{enumerate}


\end{document}