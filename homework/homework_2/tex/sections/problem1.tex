\documentclass[../main.tex]{subfiles}

\begin{document}

\problem{1}

The final Reynolds Transport Theorem we derived in class looked like:

\begin{equation*}
    \frac{\dd B_{sys}}{\dd t} = %
    \frac{\dd (mb)_{sys}}{\dd t} = %
    \pdv{t} \int_{CV} \rho b \dd \volume +%
    \int_{CS,out} b \rho \lvert{\vec{V_n}}\rvert \dd A -%
    \int_{CS,out} b \rho \lvert{\vec{V_n}}\rvert \dd A
\end{equation*}

\begin{enumerate}[label = (\alph*)]

    \item In your own words, describe what each of the three terms on the right-hand-side of the equation mean related to an arbitrary fluid extensive property, $B$.

        \begin{itemize}

            \item The first term on the RHS of the equation deals with the time rate change of some extensive property, $B$, expressed in terms of its intensive version, $b = B/m$.
            This is specifically the rate of change of the property within some control volume (CV) that we have arbitrarily defined for the purposes of analyzing a problem.
            In a steady-state problem, this term is 0, as there will be no dependency on time in that case.
            \item The second term on the RHS of the equation treats the flux of the intensive property $b$ \textit{out} of the CV across every control surface (CS).
            The velocity term is specifically the velocity normal to each CS, in order to capture the convective effect of a fluid's motion through a CV.
            \item The third term on the RHS of the equation is identical to the second term except instead of treating flux of $b$ \textit{out} of the CV, it treats flux of $b$ \textit{into} the CV.
            As before, the velocity term present in this term is specifically the velocity normal to each CS. 
            The sign convention for the fluxes is simplified in this casting of the Reynolds Transport Theorem (RTT).

        \end{itemize}

    \item If our problem was in the $x-y-z$ space, how would you represent the integrals \(\int_{CV} \dd V\) and \(\int_{CS} \dd A\) in terms of triple and double integrals, respectively?
    
    In $x-y-z$ space, the integrals can be expressed as follows:

    \[\int_{CV} \dd \volume \rightarrow \int_x \int_y \int_z \dd x \dd y \dd z\]

    \[\int_{CS} \dd A \rightarrow \int_x \int_y \dd x \dd y\]

    \textit{Note: The two specific dimensions in the second integral will vary depending on the orientation of the control surfaces relative to the major axes.}
        
    \item Why are the last two terms integral terms?

    The last two terms are integral terms because the fluid density and velocity can vary across a CS.
    By integrating across the CS, the full behavior at the boundaries can be captured.
    If there is no variation in any of the properties in either spatial dimension making up $\dd A$, there is no need for an integral. 

    \item What does the subscript ``n'' mean for the last two terms? Why do we need that there?
    
    The subscript ``n'' indicates that the velocities are \textbf{normal} to the CS.
    In this casting of the RTT we sidestep the need for vector calculus and confusing sign conventions by simply calling for the normal velocity magnitude.
    Polarity of the terms are simply defined by whether a flux is ``in'' or ``out'' of the CS.

    \item Why do we need the absolute magnitude signs around the \(\vec{V}_n\) terms?
   
    The absolute magnitude signs around the velocity terms are required for a similar reason as the ``n'' subscript: to simplify the handling of sign conventions.
    Taking the absolute magnitude of velocity removes the complexity of juggling multiple conflicting signs between velocity vectors, magnitudes, and CS normal vectors. 

    \item Why is the derivative with-respect-to \(t\) a partial derivative?
   
    The derivative w.r.t $t$ is a partial derivative because the quantities inside the partial derivative do not only vary in time --- they can also vary spatially.
    Isolating the time-variant component of the intensive property $b$ inside the CV helps capture the generation term required for proper bookkeeping of the extensive property $B$. 

    \item Explain to a classmate how our 
        \[
            \int_{CS,out} b \rho \lvert{\vec{V_n}}\rvert \dd A -%
            \int_{CS,out} b \rho \lvert{\vec{V_n}}\rvert \dd A
        \]
        term is equivalent to

        \[
            \int_{CS} b \rho \boldsymbol{V} \cdot \hat{\boldsymbol{n}} \dd A \, ,
        \]

        which is equivalent to

        \[
            \int_{CS} b \rho \vec{V} \cdot \dd \vec{A} \, .
        \]

        Be sure to explain the different math concepts. You may find it easier to ``explain'' by using a simple control-volume problem as an illustration.
        
        The second and third terms of the RHS of RTT are a decomposition of a term defined using vector notation.
        Beginning with the third integral form, we isolate the term \(\vec{V} \cdot \dd \vec{A}\).
        The mathematical meaning of \(\vec{V}\) is the fluid's velocity vector in component form (i.e., \(\hat{i}, \hat{j}, \hat{k}\) form).
        Similarly, \(\dd \vec{A}\) is the vector notation for a differential area \textit{including its normal direction}.
        As with any other vector, \(\dd \vec{A}\) can be expressed in terms of a magnitude and direction.
        For the differential area, the magnitude is simply \(\dd A\), and the normal direction can be generically expressed as \(\hat{\mathbf{n}}\).
        We can rewrite \(\dd \vec{A}\) as \(\hat{\mathbf{n}} \dd A\), as seen in the second integral form.
        The vector dot product in the third integral now becomes the dot product of the velocity vector and the differential area's normal vector, i.e., \(\mathbf{V} \cdot \hat{\mathbf{n}}\).

        Isolating the term \(\mathbf{V} \cdot \hat{\mathbf{n}}\), we recall the definition of a dot product and scalar projection. 
        The dot product is defined in euclidean space as \(\vec{a} \cdot \vec{b} = \norm{a}\norm{b}\cos{\theta}\).
        The scalar projection is defined in euclidean space as \(\mathbf{a} \cdot \hat{\mathbf{b}} = \norm{a} \cos\theta\).
        A scalar projection is a unique case of a dot product where one of the vector terms is a unit vector, denoted by \(\hat{\mathbf{v}}\).
        The result of the scalar projection is the magnitude of the vector \(\mathbf{a}\) in the direction of the unit vector \(\hat{\mathbf{b}}\).
        Replacing \(\hat{\mathbf{b}}\) with \(\hat{\mathbf{n}}\), we see that the output of the scalar product \(\mathbf{V} \cdot \hat{\mathbf{n}}\) is the magnitude of the velocity vector projected onto the normal vector of the CS, which can be expressed as \(\pm\lvert\vec{V}_n\rvert\), depending on the vector orientation.

        To understand the polarity associated with the scalar projection, we recall the behavior of $\cos{\theta}$ on $\left[{0, 2\pi}\right]$ to obtain the following ``boundary conditions'' for the function:

        \begin{align*}
            \cos{(0)} &= 1\\
            \cos{(\pi/2)} &= 0\\
            \cos{(\pi)} &= -1\\
            \cos{(3\pi/2)} &= 0
        \end{align*}

        Examining these four conditions inform us about the importance of understanding the convective vector's direction relative to a CS.
        If the velocity vector is flowing \textit{in} to the CS it will be oriented opposite the normal vector defining the CS (outward by convention), making the angle between velocity and normal vector \(\theta = \pi\).
        With this and the behavior of the $\cos$ function in mind, we see that \(\mathbf{V} \cdot \hat{\mathbf{n}}\) for an influx will have a negative polarity.
        Following the same logic for an outflux, where the velocity vector and CS normal vector make an angle of \(\theta = 0\), we see that an outflux will have a positive polarity.

        \textit{Note: another interesting observation from the behavior of $\cos()$ is that perpendicular vectors, with \(\theta =  \pi n/2\), result in 0 flux across a CS.}

        We now have all of the building blocks required to connect the three integral forms.
        We have shown that

        \[\vec{V} \cdot \dd \vec{A}\]

        is equivalent to
        \[\mathbf{V} \cdot \hat{\mathbf{n}} \dd A\,,\]

        and that 

        \[\mathbf{V} \cdot \hat{\mathbf{n}} = \pm \lvert{\vec{V}_n}\rvert\]

        depending on the vector orientation.

        Next, we have proven that velocity vectors flowing \textit{into} a CS will have a negative polarity, and velocity vectors flowing \textit{out of} a CS will have a positive polarity.
        
        \begin{align*}
        cos(0) &= 1\\
        cos(\pi) &= -1
        \end{align*}

        Therefore, 

        \begin{align*}
            \mathbf{V}_{inflow} \cdot \hat{\mathbf{n}} &= - \lvert{\vec{V}_n}\rvert\\\
            \mathbf{V}_{outflow} \cdot \hat{\mathbf{n}} &= + \lvert{\vec{V}_n}\rvert\
        \end{align*}

        Thus,

        \[
            \int_{CS} b \rho \vec{V} \cdot \dd \vec{A}
            =
            \int_{CS,out} b \rho \lvert{\vec{V_n}}\rvert \dd A -%
            \int_{CS,out} b \rho \lvert{\vec{V_n}}\rvert \dd A
        \]
        

\end{enumerate}


\end{document}