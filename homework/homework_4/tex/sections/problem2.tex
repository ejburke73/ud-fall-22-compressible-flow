\documentclass[../main.tex]{subfiles}

\begin{document}

\problem{2}

\givens{}
\(M_\infty = 3.0\)\\
\(T_\infty = 217 \,\unit{\kelvin}\)\\
\(p_\infty = 20 \,\unit{\kilo\pascal}\)\\
\(\gamma=1.4\)\\
\(R = 287\,\unit{\joule/\kilogram\,\kelvin}\)\\
\(c_p = 1000 \, \unit{\joule/\kilogram\,\kelvin}\)\\
\(q_{combustor} = 500 \, \unit{\kilo\joule/\kilogram}\)

\assumptions{}


\problempart{a} 

Briefly describe how a ramjet works.

\discussion{}
A ramjet is a form of high-speed propulsion system that does not involve any turbomachinery to compress the flow.
Only capable of starting at higher Mach numbers (~3+), a ramjet utilizes a series of oblique and normal shocks (depending on the inlet design) to compress and slow the freestream flow that is being captured for use.
The flow must be slowed to subsonic velocities for ramjet operation, as by definition a ramjet utilizes subsonic combustion.
As in a turbofan or turbojet engine, fuel is injected into the flow, combusted to generate a large pressure and temperature rise, and then accelerated out of a nozzle to convert pressure and thermal energy into kinetic energy that propels a vehicle forward.
Traditional turbomachinery runs into operational issues at high-Mach conditions due to shockwaves generated in the compression section of the flowpath.

\problempart{b} 

A simplified ramjet cycle efficiency is given by:

\[
    \eta = 1 - \left({
    \left({\frac{p_{1,\infty}}{p_{3,\infty}}}\right)^{\frac{\gamma-1}{\gamma}}
    \frac{
        \left({T_{4,\infty} - \left({\frac{p_{3,0}}{p_{1,0}}}\right)^{{\frac{\gamma-1}{\gamma}}}} \cdot T_{3,\infty}\right)
    }{\left({T_{4,\infty}-T_{3,\infty}}\right)}
    }\right)
\]

Use this equation to come up with the optimal spike half angle (to the nearest degree) for the given cruise conditions. 
You must write out your general methodology for the grader.
Include a plot of the efficiency versus spike half angle.

\givens{}

\assumptions{}

\solution{}

\problempart{c} 

What is the efficiency of the ramjet if we get rid of the spike altogether?

\givens{}

\assumptions{}

\solution{}

\problempart{d} 

Write up a description of your observations of the ramjet with and without the spike.
Be sure to include relevant compressible-flow theories/jargon. 
Be sure to include a conversation of the effect of inlet freestream pressure ratio, inlet total pressure ratio, and combustor freestream temperature difference.

\givens{}

\assumptions{}

\solution{}

\problempart{e} 

For the optimal spike half angle solved for in part (b), let's now explore the effect of cruise Mach number on efficiency. 
Plot efficiency versus cruise Mach number.

\givens{}

\assumptions{}

\solution{}

\problempart{f} 

Write up a description of your observations of the Mach-number effect on ramjets. 
Be sure to include the phenomena that limit the ramjet's performance at higher Mach numbers.
Be sure to include relevant compressible-flow theories/jargon. 
Be sure to include a conversation of the effect of inlet freestream pressure ratio, inlet total pressure ratio, and combustor freestream temperature difference.

\givens{}

\assumptions{}

\solution{}

\problempart{g} 

What do you expect to happen if we instead treat the spike as an axisymmetric cone?
Briefly justify your answer with words.

\givens{}

\assumptions{}

\solution{}

\end{document}