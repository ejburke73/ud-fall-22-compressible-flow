\documentclass[../main.tex]{subfiles}

\begin{document}

\problem{3}

\problempart{a}

Briefly describe why scramjets (theoretically) solve some of the specific issues that ramjets encounter at higher Mach numbers.

\problempart{b}

How do the inlet freestream pressure ratio, inlet total pressure ratio, and combustor inlet freestream temperature compare for this scramjet design compared to the ramjet (with spike) design? 
You must actually calculate the numbers. 
Does this support what you said in part (a)?

\problempart{c}

Under these conditions, what is the Mach number of the flow at the inlet of the combustor? 
What combustion challenges do we face in efficiently burning fuel at this Mach number?

\problempart{d}

Let's now investigate the qualitative effect of viscosity on shock reflections.
We will focus our attention to the last reflected shock before the fuel-injection site.
An effect of viscosity is to decelerate the flow in the vicinity of the wall such at \(u_{wall}=0\).
We will assume that the scramjet is a height of \(H\) from the bottom wall to the top wall.
We will also assume that u is equal to its local freestream value a distance \(\delta\) away from the walls.
Sketch \(y\) versus \(\beta\) for \(0<y<H\).
In addition, sketch the more realistic shape of the shock.

\end{document}